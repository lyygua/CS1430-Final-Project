%%%%%%%%%%%%%%%%%%%%%%%%%%%%%%%%%%%%%%%%%%%%%%%%%%%%%%%%%%%%%%%%%%%%%%%%%%%%%%%%%%%%%%%%%%%%%%%%
%
% CSCI 1430 Written Question Template
%
% This is a LaTeX document. LaTeX is a markup language for producing documents.
% Your task is to answer the questions by filling out this document, then to
% compile this into a PDF document.
%
% TO COMPILE:
% > pdflatex thisfile.tex

% If you do not have LaTeX, your options are:
% - VSCode extension: https://marketplace.visualstudio.com/items?itemName=James-Yu.latex-workshop
% - Online Tool: https://www.overleaf.com/ - most LaTeX packages are pre-installed here (e.g., \usepackage{}).
% - Personal laptops (all common OS): http://www.latex-project.org/get/ 
%
% If you need help with LaTeX, please come to office hours.
% Or, there is plenty of help online:
% https://en.wikibooks.org/wiki/LaTeX
%
% Good luck!
% The CSCI 1430 staff
%
%%%%%%%%%%%%%%%%%%%%%%%%%%%%%%%%%%%%%%%%%%%%%%%%%%%%%%%%%%%%%%%%%%%%%%%%%%%%%%%%%%%%%%%%%%%%%%%%
%
% How to include two graphics on the same line:
% 
% \includegraphics[width=0.49\linewidth]{yourgraphic1.png}
% \includegraphics[width=0.49\linewidth]{yourgraphic2.png}
%
% How to include equations:
%
% \begin{equation}
% y = mx+c
% \end{equation}
% 
%%%%%%%%%%%%%%%%%%%%%%%%%%%%%%%%%%%%%%%%%%%%%%%%%%%%%%%%%%%%%%%%%%%%%%%%%%%%%%%%%%%%%%%%%%%%%%

\documentclass[11pt]{article}

\usepackage[english]{babel}
\usepackage[utf8]{inputenc}
\usepackage[colorlinks = true,
            linkcolor = blue,
            urlcolor  = blue]{hyperref}
\usepackage[a4paper,margin=1.5in]{geometry}
\usepackage{stackengine,graphicx}
\usepackage{fancyhdr}
\setlength{\headheight}{15pt}
\usepackage{microtype}
\usepackage{times}
\usepackage{booktabs}

% From https://ctan.org/pkg/matlab-prettifier
\usepackage[numbered,framed]{matlab-prettifier}

\frenchspacing
\setlength{\parindent}{0cm} % Default is 15pt.
\setlength{\parskip}{0.3cm plus1mm minus1mm}

\pagestyle{fancy}
\fancyhf{}
\lhead{Final Project Progress Report}
\rhead{CSCI 1430}
\rfoot{\thepage}

\date{}

\title{\vspace{-1cm}Final Project Progress Report}

\begin{document}
\maketitle
\vspace{-1cm}
\thispagestyle{fancy}
**Important**: In your report, please
1) Make it very clear who on the team contributed what, and 
2) Include the dates of at least *two* meetings you had with your mentor.
\textbf{Team name: \emph{Pixel Pioneers}}\\
\textbf{TA name: \emph{Hannah}}

\emph{Note:} when submitting this document to Gradescope, make sure to add all other team members to the submission. This can be done on the submission page after uploading.

\section*{Progress Report Instructions}

Before writing your progress report, you should have met with your TA and talked through your progress.

\subsection*{Team contributions}

Please describe in one paragraph (3--4 sentences) per team member what each of you contributed to the project so far.
\begin{description}
\item[Person 1] Completed a comprehensive review of over 20 research papers and articles focusing on adversarial attack methods and their transferability across models; Identified key techniques that enhance transferability, including gradient-based methods and decision boundary analysis; Presented a summary of defense mechanisms and their effectiveness against state-of-the-art adversarial attacks.
\item[Person 2] Through research, confirm the types of popular neural network models. Then, write code to build and run part of the neural network models for image classification tasks. Collect a sufficient amount of image samples from different categories, and preprocess these images. Use the neural network to conduct a preliminary validation of the classification effectiveness on these unattacked images.
\item[Person 3] I went through a significant amount of literature to learn about different transfer attack methods. This included digging into gradient-based attacks, which are pretty common, and also checking out input transformation-based attacks. Along the way, I also summarized some of the newer algorithms designed to create adversarial examples. This deep dive has given me a thorough understanding of how our model is structured and has helped me identify which attack methods might be most effective for us to incorporate.

\end{description}

\end{document}